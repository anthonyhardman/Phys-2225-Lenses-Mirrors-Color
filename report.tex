% !TEX program = xelatex
\documentclass{report}
\usepackage{listings}
\usepackage{amsmath}
\usepackage{xfrac}
\usepackage{amsfonts}
\usepackage{multirow}
\usepackage{pgfplots}
\usepackage{geometry}
\usepackage{hhline}
\usepackage{fontspec}
\usepackage{graphicx}
\usepackage{float}
\usepackage{tabularx}
\setmonofont{Consolas}
\geometry{legalpaper, portrait, margin=0.50in}
\pgfplotsset{compat=newest}
\usetikzlibrary{pgfplots.groupplots}
\usepackage{inconsolata}
\renewcommand*\familydefault{\ttdefault} %% Only if the base font of the document is to be typewriter style
\usepackage[T1]{fontenc}
\begin{document}
    \begin{center}
        \section*{PHYS 2225}
        \section*{Lenses, Mirrors, Color}
        \subsection*{Anthony Hardman}
        \subsubsection*{with lab partners Garrett Vernon and Bridger Thompson}
    \end{center}
    \section*{Objective:}
    \paragraph{(1) Use a converging lens to form a real and virtual images and to compare the measured
    position and magnification of each image with theory; (2) Construct some optical instruments,
    measure their magnification, and compare with theory; (3) investigate concave and convex mirrors;
    (4) Observe the characteristics of colors produced by additive and subtractive mixing of light.}

    \section*{Theory:}
    The three principle rays when working with lenses:
    \begin{enumerate}
        \item Parallel ray: Drawn parallel to the surface that the lens is normal to from the top of the object. This ray bends at the lens 
        plane and goes through the focal point.
        \item Central Ray: Drawn through the center of the lens from the top of the object, does not bend at the lens plane.
        \item Focal point ray: Drawn from the secondary focal point, through the top of the object, and to the lens plane. This ray bends parallel
        at the lens plane.
    \end{enumerate}
    From these rays we can determine the real or virtual image that results from refraction through a lens.
    \\ \\
    These equations will be used to calculate theoretical values so that we can compare them to values measured during procedures:
    \begin{equation} \label{eq:thinslens}
        \dfrac{1}{s} + \dfrac{1}{s^\prime} = \dfrac{1}{f}
    \end{equation}
    The thin lens equation, where $s$ is the distance of the object from the lens plane, $s^\prime$ is the distance of the 
    image from the lens plane, ad $f$ is the focal length of the lense.

    \begin{equation} \label{eq:magnification}
        m = -\dfrac{s^\prime}{s} = \dfrac{h^\prime}{h}
    \end{equation}
    The magnification of an image, where $m$ is the magnification of the image, $h^\prime$ is the height of the image, and $h$ 
    is the height of the object.

    \section*{Procedures for one lens:}
    \begin{enumerate}
        \item 
    \end{enumerate}
    \subsection*{Data:}
    \section*{Procedures for multi-lens optical instruments:}
    \begin{enumerate}
        \item 
    \end{enumerate}
    \subsection*{Data:}
    \section*{Procedures for mirrors:}
    \begin{enumerate}
        \item 
    \end{enumerate}
    \subsection*{Data:}
    \section*{Procedures for color mixing:}
    \subsection*{Part A - Additive Mixing:}
    \begin{enumerate}
        \item 
    \end{enumerate}
    \subsubsection*{Data:}
    \subsection*{Part B - Substractive mixing:}
    \begin{enumerate}
        \item 
    \end{enumerate}
    \subsubsection*{Data:}

    \section*{Questions:}
    \begin{enumerate}
        \item 
    \end{enumerate}
    \section*{Conclusions:}
\end{document}